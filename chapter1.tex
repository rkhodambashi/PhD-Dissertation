%\graphicspath{Images/Chap1/}
\chapter{introduction}
\label{chap:intro}

%\def{pdf}
%\ifpdf
    %\graphicspath{{Chapter1/Figs/Raster/}{Chapter1/Figs/PDF/}{Chapter1/Figs/}}
%\else
    %\graphicspath{{Chapter1/Figs/Vector/}{Chapter1/Figs/}}
%\fi
Traditionally, robots are made of rigid materials such as steel. Rigidity enables the robots to perform repeated tasks with high speed and precision. The downside is that they are usually designed for a specific task. Biological organisms on the other hand, can perform a wide range of tasks but with lower precision and speed. Therefore, scientists have been inclined towards using soft materials in robots in order to equip them with some of the features seen in soft living organisms. In this chapter, rigid robots are contrasted against soft living organisms in terms of their capabilities. Then some challenges are listed and the ones addressed by this thesis are discussed. 

\section{Soft Robotics as an Emerging Field}
\label{sec:emerging}

Majority of biological organisms contain soft tissue. The soft tissue brings some advantages to these organisms. Octopus is the most widely used source of inspiration for soft robotic research. Octopuses can deform their body and pass through small openings. Therefore, shape morphing is one of the key advantages of soft organisms. Octopuses can also soften their arm as they wrap it around a bottle cap and stiffen their arms for a tight grip as they turn the cap to open it as shown in Fig. Variable stiffness is therefore, another aspect of soft organisms. Human hand is another example which can demonstrate some of the advantages of compliant materials. Human hand can grasp objects with a wide range of shapes and surface roughness without slipping because the soft tissue passively conforms to different shapes (Fig). Human hand can also absorb energy from an impact such as catching a baseball, preventing any damages to the hand. Impact tolerance is another key advantages of soft bodied organisms. The ability to absorb the impact energy also protects the surrounding objects and humans in case of a collision. This brings up safety as another important feature of compliance present in soft bodied animals. The advantages of soft animals are summarized in Table. Inspired by biology, soft robot developers try to utilize the advantages inherent in soft, compliant matter to achieve safer interactions around humans or more robust locomotion and manipulation in unstructured environments~\cite{martinez2013robotic,laschi2012soft,tolley2014resilient,bilodeau2015monolithic}.

\section{State of the Art in Soft Robotics}
While traditionally robots were unaware of their surroundings and operated based on a prescribed set of instructions, recent advances in sensor technologies have paved the road to more autonomous robots. Therefore, in addition to actuators, sensors and computers have also become and integral part of the soft robots. For a soft robot, ideally all these components should be soft. Here, the state of the art technology in each category is reviewed with a special attention paid to the soft actuators which is the focus of this dissertation. Soft robots are categorized based on many different features. Here, we focus on miniature and untethered robot categorization to stay focused on the topic of this thesis. Since manufacturing is part of the contribution of this thesis, a brief survey of the manufacturing techniques is also presented.
%\subsection{Components of a Soft Robot}
\subsection{Soft Sensors}
\subsection{Soft Computers}
\subsection{Soft Actuators}
Soft pneumatic actuators (SPAs) \cite{Gorissen2017, branyan2017soft} are the most widely used category in soft robotics. SPAs use passive materials such as silicone and rely on rigid components such as motors and pumps that are difficult to downscale and therefore, manufacturing small-scale soft actuators which have applications as envisioned by \cite{hines2017soft} has remained a bottleneck in the development of miniaturized soft robots \cite{majidi2019soft}. 
\subsection{Untethered Soft Robots}
\subsection{Miniature Soft Robots}
\subsection{Miniature Untethered Soft Robots}
\subsection{Soft Robot Manufacturing Techniques}
\subsubsection{Molding}
\subsubsection{3D Printing}
\section{Challenges Ahead}
\subsection{Miniaturization Challenges}
\subsection{Manufacturing Challenges}
\section{Contributions of this Dissertation}
\subsection{Broader Impact}
\subsubsection{Scientific}
Current soft actuators rely on additional hardware such as pumps, high voltage supplies, light generation sources and magnetic field generators for their operation. These components resist miniaturization and embedding them into small-scale soft robots would be challenging. This limits their mobile applications especially in hyper-redundant robots where a high number of actuators are needed. On the other hand, our developed SVAs weigh only 100 mg and require small footprint microcontrollers for their operation which can be embedded in the robotic system. 
\subsubsection{Social}
Soft robots are intrinsically safe around humans. This can help bring robots to daily life applications such as household robots and assistant robots for elderly people. Also, our soft robots can operate underwater due to hydrogel compatibility with moist environments. A swarm of soft robots can be used for underwater exploration and data collection and help monitor the climate change through recording the water current temperatures over long period of time.

\subsection{Dissertation Outline}



