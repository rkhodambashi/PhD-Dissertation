\chapter{conclusions and future work}
\label{chap:conclusion}
This Ph.D. research is performed as part of ONR.

scientific impact
	Impact on material science
		the research has helped better understand the material
	Impact on robotics
		methods for miniature untethered robots


Broader impact
Soft robotics has quickly turned into a broad area of research since its introduction. Therefore, new challenges have already been raised and more will be faced in the future. One of these challenges is to cut the tether from robots and enable them to operate as mobile devices. Another challenge is to reduce the size of the robots. An even tougher problem would arise when trying to solve both the aforementioned challenges simultaneously. Addressing these challenges require innovative approaches and usually requires progress in multiple fields. However, often the progress in one field is made without considering the limitations of the other fields. This problem is more profound in the field of material science regarding the development on novel materials for soft actuators and sensors. For instance, stimuli-responsive hydrogels are usually developed using complex processes and specialized equipment and therefore they are less accessible to robotics community. 
SVAs can be mass produced and used as off-the-shelf components 

Impact on community





In this dissertation, enabling technologies and simultaneous innovations in the areas of material synthesis, processing and manufacturing  is introduced as opposed to other research which usually focus only on one of the areas. These innovations resulted in a hydrogel that has fast response, tunable response, rapid polymerization time (under 15 seconds) and uses cheap and readily available ingredients (DI water). While in the prior work we surveyed, only some of the features have been achieved at a time. We have combined these features with voxel-based manufacturing, which facilitates material distribution and placement of actuators. Although we have used hydrogels for producing the voxels, the range of materials that can be used using this method is wider compared to 3D printing or lithography methods. These contributions are manifested through our demonstrations which present a class of heterogeneous hydrogel structures with unique capabilities not demonstrated previously in the literature. These capabilities are on-demand dynamic shape changes which allows the structures to interact with unstructured environments. In other words, without these innovations, the demonstrations would not be possible. Therefore, we think our proposed methods which are developed through cross functional collaboration of material scientists and roboticists satisfy multiple requirements that a material needs to have to be widely accepted for soft robotics applications. This can be considered as our main contribution on top of others. 

Future