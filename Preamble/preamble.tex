%\usepackage{natbib}
\usepackage{geometry}
\usepackage{fancyhdr}
\usepackage{afterpage}
\usepackage{graphicx}
\usepackage{amsmath,amssymb,amsbsy}
\usepackage{dcolumn,array}
\usepackage{tocloft}
\usepackage{asudis}
\usepackage{pdfpages}
\usepackage{subfiles}
\usepackage{gensymb}
\usepackage{xcolor}
\usepackage{soul}
%\usepackage{ulem} 
\usepackage{array}
\usepackage{amsfonts}
\usepackage{times}
\usepackage{algorithm}
\usepackage[noend]{algpseudocode}
\usepackage[normalem]{ulem}
\usepackage{amsmath}
\usepackage{amsmath,amssymb,amsbsy}
\usepackage{commath}
\usepackage{breqn}
\usepackage{enumerate}
\usepackage{epstopdf}
\usepackage{textcomp}
\usepackage{balance}
\usepackage[utf8]{inputenc}
\usepackage[T1]{fontenc}
%\usepackage[pageanchor=true,plainpages=false,pdfpagelabels=true,pagebackref=true,bookmarks=true,bookmarksnumbered=true,bookmarksopen=true]{hyperref}
\usepackage[hidelinks]{hyperref}

%\usepackage[pageanchor=true,plainpages=false,pdfpagelabels,bookmarks,bookmarksnumbered,bookmarksopen=true]{hyperref}
\usepackage{bookmark}
\newcommand{\vect}[1]{\boldsymbol{#1}}
\newcommand{\figref}[1]{Figure \ref{#1}}
\newcommand{\firstfigref}[1]{\textbf{Figure} \ref{#1}}
\newcommand{\subfigref}[2]{Figure \ref{#1}-{#2}}
% \newcommand{\firstsubfigref}[2]{\textbf{Figure} \textbf{\ref{#1}}-{#2}}
\newcommand{\firstsubfigref}[2]{\textbf{Figure \ref{#1}}-{#2}}
\newcommand{\citesuperscript}[1]{\cite{#1}}
%% ******************************************************************************
%% ****************************** Custom Margin *********************************
%
%% Add `custommargin' in the document class options to use this section
%% Set {innerside margin / outerside margin / topmargin / bottom margin}  and
%% other page dimensions
%%\ifsetCustomMargin
  %%\RequirePackage[left=37mm,right=30mm,top=35mm,bottom=30mm]{geometry}
  %%\setFancyHdr % To apply fancy header after geometry package is loaded
%%\fi
%% Add `custommargin' in the document class options to use this section
%% Set {innerside margin / outerside margin / topmargin / bottom margin}  and
%% other page dimensions
%%\ifsetCustomMargin
  %%\RequirePackage[left=37mm,right=30mm,top=35mm,bottom=30mm]{geometry}
  %%\setFancyHdr % To apply fancy header after geometry package is loaded
%%\fi
%
%\usepackage{geometry}
%\usepackage{fancyhdr}
%% *****************************************************************************
%% ******************* Fonts (like different typewriter fonts etc.)*************
%
%% Add `customfont' in the document class option to use this section
%
%%\ifsetCustomFont
  %%% Set your custom font here and use `customfont' in options. Leave empty to
  %%% load computer modern font (default LaTeX font).
  %%\RequirePackage{helvet}
%%\fi
%
%% *****************************************************************************
%% **************************** Custom Packages ********************************
%
%% ************************* Algorithms and Pseudocode **************************
%
%%\usepackage{algpseudocode}
%
%
%% ********************Captions and Hyperreferencing / URL **********************
%
%% Captions: This makes captions of figures use a boldfaced small font.
\RequirePackage[small,bf]{caption}
%
\RequirePackage[labelsep=space,tableposition=top]{caption}
%\renewcommand{\figurename}{Fig.} %to support older versions of captions.sty
%
%
%% *************************** Graphics and figures *****************************
%
%%\usepackage{rotating}
%%\usepackage{wrapfig}
%
%% Uncomment the following two lines to force Latex to place the figure.
%% Use [H] when including graphics. Note 'H' instead of 'h'
%%\usepackage{float}
%%\restylefloat{figure}
%
%% Subcaption package is also available in the sty folder you can use that by
%% uncommenting the following line
%% This is for people stuck with older versions of texlive
\usepackage{subcaption}
%\usepackage{caption}
%
%% ********************************** Tables ************************************
%\usepackage{booktabs} % For professional looking tables
%\usepackage{multirow}
%
%%\usepackage{multicol}
%%\usepackage{longtable}
%%\usepackage{tabularx}
%
%
%% ***************************** Math and SI Units ******************************
%\usepackage[acronym]{glossaries}
%\usepackage{amsfonts}
%\usepackage{amsmath}
%\usepackage{amssymb}
%\usepackage{siunitx} % use this package module for SI units
%\usepackage[parfill]{parskip}
%\usepackage{algpseudocode}
%\usepackage{algorithm} %ctan.org\pkg\algorithms
%\usepackage{hyperref}
%
\usepackage{graphics} % for pdf, bitmapped graphics files
\usepackage{epsfig} % for postscript graphics files
%\usepackage{mathptmx} % assumes new font selection scheme installed
%\usepackage{times} % assumes new font selection scheme installed
%\usepackage{mathtools}
%\usepackage[noadjust]{cite}
%\usepackage{dblfloatfix}
%\usepackage{color}
\usepackage{booktabs,tabularx}
%\usepackage{optidef}
\usepackage[font=small,skip=3pt]{caption}
%\usepackage{dcolumn,array}
%\usepackage{tocloft}
%\usepackage{asudis}
%\usepackage[pageanchor=true,plainpages=false,pdfpagelabels,bookmarks,bookmarksnumbered]{hyperref}
%
%%\usepackage{acro}
%
%% ******************************* Line Spacing *********************************
\usepackage{listings}

%\definecolor{mygreen}{rgb}{0,0.6,0}
%\definecolor{mygray}{rgb}{0.5,0.5,0.5}
%\definecolor{mymauve}{rgb}{0.58,0,0.82}

\lstset{ %
  backgroundcolor=\color{white},   % choose the background color; you must add \usepackage{color} or \usepackage{xcolor}
  basicstyle=\footnotesize,        % the size of the fonts that are used for the code
  breakatwhitespace=false,         % sets if automatic breaks should only happen at whitespace
  breaklines=true,                 % sets automatic line breaking
  captionpos=b,                    % sets the caption-position to bottom
  commentstyle=\color{mygreen},    % comment style
  deletekeywords={...},            % if you want to delete keywords from the given language
  escapeinside={\%*}{*)},          % if you want to add LaTeX within your code
  extendedchars=true,              % lets you use non-ASCII characters; for 8-bits encodings only, does not work with UTF-8
  frame=single,	                   % adds a frame around the code
  keepspaces=true,                 % keeps spaces in text, useful for keeping indentation of code (possibly needs columns=flexible)
  keywordstyle=\color{blue},       % keyword style
  language=Octave,                 % the language of the code
  otherkeywords={*,...},            % if you want to add more keywords to the set
  numbers=left,                    % where to put the line-numbers; possible values are (none, left, right)
  numbersep=5pt,                   % how far the line-numbers are from the code
  numberstyle=\tiny\color{mygray}, % the style that is used for the line-numbers
  rulecolor=\color{black},         % if not set, the frame-color may be changed on line-breaks within not-black text (e.g. comments (green here))
  showspaces=false,                % show spaces everywhere adding particular underscores; it overrides 'showstringspaces'
  showstringspaces=false,          % underline spaces within strings only
  showtabs=false,                  % show tabs within strings adding particular underscores
  stepnumber=2,                    % the step between two line-numbers. If it's 1, each line will be numbered
  stringstyle=\color{mymauve},     % string literal style
  tabsize=2,	                   % sets default tabsize to 2 spaces
  title=\lstname                   % show the filename of files included with \lstinputlisting; also try caption instead of title
}

%
%\usepackage{calc}
%\usepackage{graphicx}
%\usepackage{xcolor}
%\usepackage{soul}
%\usepackage{afterpage}
%
%% Choose linespacing as appropriate. Default is one-half line spacing as per the
%% University guidelines
%
%% \doublespacing
%% \onehalfspacing
%% \singlespacing
%
%
%% ************************ Formatting / Footnote *******************************
%
%% Don't break enumeration (etc.) across pages in an ugly manner (default 10000)
%%\clubpenalty=500
%%\widowpenalty=500
%
%%\usepackage[perpage]{footmisc} %Range of footnote options
%
%
%% *****************************************************************************
%% *************************** Bibliography  and References ********************
%
%%\usepackage{cleveref} %Referencing without need to explicitly state fig /table
%
%%% Add `custombib' in the document class option to use this section
%\ifuseCustomBib
\RequirePackage[square, sort, numbers]{natbib} % CustomBib
%\usepackage{natbib}
%% If you would like to use biblatex for your reference management, as opposed to the default `natbibpackage` pass the option `custombib` in the document class. Comment out the previous line to make sure you don't load the natbib package. Uncomment the following lines and specify the location of references.bib file
%
%%\RequirePackage[backend=biber, style=numeric-comp, citestyle=numeric, sorting=nty, natbib=true]{biblatex}
%%\bibliography{References/references} %Location of references.bib only for biblatex
%
%%\fi
%%
%%% changes the default name `Bibliography` -> `References'
%%\renewcommand{\bibname}{References}
%
%
%% *****************************************************************************
%% *************** Changing the Visual Style of Chapter Headings ***************
%% This section on visual style is from https://github.com/cambridge/thesis
%
%% Uncomment the section below. Requires titlesec package.
%
%%\RequirePackage{titlesec}
%%\newcommand{\PreContentTitleFormat}{\titleformat{\chapter}[display]{\scshape\Large}
%%{\Large\filleft{\chaptertitlename} \Huge\thechapter}
%%{1ex}{}
%%[\vspace{1ex}\titlerule]}
%%\newcommand{\ContentTitleFormat}{\titleformat{\chapter}[display]{\scshape\huge}
%%{\Large\filleft{\chaptertitlename} \Huge\thechapter}{1ex}
%%{\titlerule\vspace{1ex}\filright}
%%[\vspace{1ex}\titlerule]}
%%\newcommand{\PostContentTitleFormat}{\PreContentTitleFormat}
%%\PreContentTitleFormat
%
%
%% ******************************************************************************
%% ************************* User Defined Commands ******************************
%% ******************************************************************************
%
%% *********** To change the name of Table of Contents / LOF and LOT ************
%
%%\renewcommand{\contentsname}{My Table of Contents}
%%\renewcommand{\listfigurename}{My List of Figures}
%%\renewcommand{\listtablename}{My List of Tables}
%
%
%% ********************** TOC depth and numbering depth *************************
%
\setcounter{secnumdepth}{4}
\setcounter{tocdepth}{4}
%
%%\makeglossaries
%
%% ******************************* Nomenclature *********************************
%
%% To change the name of the Nomenclature section, uncomment the following line
%
%%\renewcommand{\nomname}{Symbols}
%
%
%% ********************************* Appendix ***********************************
%
%% The default value of both \appendixtocname and \appendixpagename is `Appendices'. These names can all be changed via:
%
%%\renewcommand{\appendixtocname}{List of appendices}
%%\renewcommand{\appendixname}{Appndx}
%
%% ******************************** Draft Mode **********************************
%
%% Uncomment to disable figures in `draftmode'
%%\setkeys{Gin}{draft=true}  % set draft to false to enable figures in `draft'
%
%% These options are active only during the draft mode
%% Default text is "Draft"
%%\SetDraftText{DRAFT}
%
%% Default Watermark location is top. Location (top/bottom)
%%\SetDraftWMPosition{bottom}
%
%% Draft Version - default is v1.0
%%\SetDraftVersion{v1.1}
%
%% Draft Text grayscale value (should be between 0-black and 1-white)
%% Default value is 0.75
%%\SetDraftGrayScale{0.8}
%
%
%%% Todo notes functionality
%%% Uncomment the following lines to have todonotes.
%
%%\ifsetDraft
%%	\usepackage[colorinlistoftodos]{todonotes}
%%	\newcommand{\mynote}[1]{\todo[author=kks32,size=\small,inline,color=green!40]{#1}}
%%\else
%%	\newcommand{\mynote}[1]{}
%%	\newcommand{\listoftodos}{}
%%\fi
%
%% Example todo: \mynote{Hey! I have a note}
