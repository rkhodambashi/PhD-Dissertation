\begin{abstract}


Current  soft  robots  rely  on  additional  hard-ware  such  as  pumps,  high  voltage  supplies,  light  generation sources   and   magnetic   field   generators   for   their   operation. These  components  resist  miniaturization  and  embedding  them into  small-scale  soft  robots  is challenging.  This  limits their  mobile  applications  where  the  entire  system  needs  to be  untethered  especially  in  hyper-redundant  robots  where  a high  number  of  actuators  are  needed. This dissertation aims at addressing some of the challenges associated with creating miniature, untethered soft robots that can function without any attachment to external power supplies or receiving any control signals from outside sources. This goal is accomplished by introducing a soft active material and a manufacturing method which together, facilitate the miniaturization of soft robots and effectively supports their autonomous, mobile operation without any connection to outside equipment or human intervention. 

The soft active material presented here is a poly(N-isopropylacrylamide) (PNIPAAm) based hydrogel. This hydrogel respond to changes in the surrounding temperature and responds by changing its volume. The volume change is a result of water transport in the porous structure of the hydrogel. The polymer chains switch from hydrophilic to hydrophobic when the temperature rises above their transition temperature and therefore, water is removed from the polymer network resulting in collapse of the network and a reduction of volume. A major challenge regarding PNIPAAm-based hydrogels is their slow response making them unsuitable for robotic applications. This challenge is addressed by introducing a mixed-slolvent photopolymerization technique which alters the pore strecuture of the hydrogel and facilitates the transport of water and thus, the rate of volume change. Using this technique, the re-swelling response time of hydrogels is reduced to 2.4 min which is over 25 times faster than previously demonstrated hydrogels. The mixed-solvent method also provides a means of tuning the material properties of hydrogels including its response rate and Young's modulus by controlling the solvent ratio. The one-step photopolymerization using UV light is performed in under 15 sec which is a significant improvement over thermopolymerization which takes a few minutes to several hours. The fast polymerization time is critical in fabrication of soft robot prototypes in a timely manner. 

To address the challenges associated with manufacturing, a new category of soft actuators called soft voxel actuators (SVAs) is presented. SVAs are active  voxels (volumetric pixels) made  using  stimuli-responsive  hydrogels. In designing the SVAs, the analogy between electrical activation of muscle tissue by nervous system in animals is used: SVAs are actuated  by  electrical  currents  through  Joule  heating.  SVAs weighing  only  100 mg  require  small  footprint  microcontrollers for  their  operation  which  can  be  embedded  in  the  robotic system. SVAs can be considered as building blocks  assembled to form complex robots which can perform tasks that require high number of degrees of freedom (DOF). The  advantages  of hydrogel-based SVAs are demonstrated through different robotic platforms namely a  hyper-redundant  manipulator  with 16 SVAs and an untethered miniature robot for mobile underwater applications using 8 SVAs.


\end{abstract}