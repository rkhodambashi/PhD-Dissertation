\begin{abstract}


Soft robots currently rely on additional hardware such as pumps,  high voltage supplies,  light generation sources, and magnetic field generators for their operation. These components resist miniaturization; thus embedding them into small-scale soft robots is challenging.  This limits their applications where the system needs to be untethered, especially in hyper-redundant mobile robots where a high number of actuators are needed. This dissertation aims at addressing some of the challenges associated with creating miniature, untethered soft robots that can function without any attachment to external power supplies or receiving any control signals from outside sources. This goal is accomplished by introducing a soft active material and a manufacturing method that together, facilitate the miniaturization of soft robots and effectively supports their autonomous, mobile operation without any connection to outside equipment or human intervention. 

The soft active material presented here is a hydrogel based on a polymer called poly(N-isopropylacrylamide) (PNIPAAm). This hydrogel responds to changes in the temperature and responds by expanding or contracting. Volumetric change is a result of water transport to and out of the porous structure of the hydrogel. The PNIPAAm chains switch from hydrophilic to hydrophobic when the temperature rises above a transition temperature; thus water is forced out from the polymer network resulting in the collapse of the network and a reduction of volume. A major challenge regarding PNIPAAm-based hydrogels is their slow response which has historically made them unsuitable for robotic applications. This challenge is addressed by introducing a mixed-solvent photo-polymerization technique that alters the pore structure of the hydrogel and facilitates the water transport and thus the rate of volume change. Using this technique, the re-swelling response time of hydrogels is reduced to 2.4 min--over 25 times faster than hydrogels demonstrated previously. The mixed-solvent method also provides a means of tuning the material properties of hydrogels including their response rate and Young's modulus by controlling the solvent ratio. The one-step photo-polymerization using UV light is performed in under 15 sec, which is a significant improvement over thermo-polymerization, which takes anywhere between a few minutes to several hours. Photopolymerization is key towards simplifying recipes, improving access to these techniques, and making them tractable for iterative design processes.

To address the challenges associated with manufacturing, a new category of soft actuators called soft voxel actuators (SVAs) is presented. SVAs are active voxels (volumetric pixels) made using stimuli-responsive hydrogels. In designing SVAs, the analogy between electrical activation of muscle tissue by the nervous system in animals is used: SVAs are actuated by electrical currents through  Joule heating.  SVAs weighing only  100 mg require small footprint microcontrollers for their operation which can be embedded in the robotic system. SVAs can be considered as building blocks assembled to form complex robots which can perform tasks that require a high number of degrees of freedom (DOF). The advantages of hydrogel-based SVAs are demonstrated through different robotic platforms namely a  hyper-redundant manipulator with 16 SVAs, an untethered miniature robot for mobile underwater applications using 8 SVAs, and a gripper using 32 SVAs.


\end{abstract}